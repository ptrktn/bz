%-----------------------------------------------------------------------------%
% ez3d_doc.tex -- EZ-Scroll documentation                                     %
%                                                                             %
% $Revision: 1.1.1.1 $
% $Date: 2003/07/05 15:08:18 $
% 
%-----------------------------------------------------------------------------%
%-----------------------------------------------------------------------------%

% This is a LaTex source file.

\documentstyle[12pt]{article}
\textheight 9in  \textwidth 6.5in
\topmargin -.7in \oddsidemargin .0in

\begin{document}
\baselineskip = 20pt plus 1pt minus 1pt
\parskip = 8pt
\parindent=0pt

\begin{center} 
{\bf EZ-SCROLL DOCUMENTATION} 
\end{center} 

\bigskip
{\bf I. General} 
\smallskip

This package uses OpenGL for 3D rendering and has been optimized for
use on SGI workstations.  By using the Mesa library (public domain
implementation of most OpenGL routines) it should be possible to run
on virtually any machine supporting X.  I routinely run EZ-Scroll on
my PC with the Linux operating system.  See
http://www.ssec.wisc.edu/$\sim$brianp/Mesa.html 
Note, you can run without graphics, but you must have OpenGL (or Mesa)
header files and libraries to use the code.

The EZ-Scroll package (in particular this document) is under
development.  There are aspects of the code which I am not happy with,
but to my knowledge everything works correctly.

The philosophy is to keep programs as simple as possible and to
provide documentation by way of comments within the code itself.  The
user is expected to modify the programs according to his or her needs.
The bulk of the package is devoted to graphics. Almost all of the
execution time is spent in a loop in the routine {\sf Step()} in {\em
ezstep3d.c}.

The computational methods are described in more detail in the
references at the end of this document.  Ref.~[5] describes the 3D
implementation.  Ref.~[2] is the original source for the model.  If
you generate publications from using EZ-Scroll, I ask that you cite
these papers.


\medskip
{\bf II. Running EZ-Scroll} 
\smallskip

{\bf Files:} You should have the following files: \\ {\em ezscroll.c,
ezstep3d.c, ezgraph3d.c, ezmarching.c, ezscroll.h, ezstep3d.h,
ezgraph3d.h, ezmarching.h, task.dat}, and {\em Makefile}. \\ You will
probably want to save copies of these files (in compressed tar
format).

{\bf make}: It is up to you to edit {\em Makefile} as necessary for
your system. You can either compile using SGI C (cc) or else using GNU
C (gcc).  You may, if you wish, specify NX etc.~at compile time.  Then
these will be ignored in the task file.

Note: On an SGI, using the SGI C compiler cc with -DNX etc gives the
fastest execution.  Using the GNU C compiler gcc with -DNX is slightly
slower and gcc without -DNX is slightly slower still.  Using cc
without -DNX is terrible and should not be used, i.e.~if you want to
specify the number of grid points through the {\em task.dat} file,
then you should to use GNU compiler and not the SGI C compiler.  This
is still being looked into.

Make {\em ezscroll} by typing {\em make}. Then run by typing {\em
ezscroll}. A window should open containing an initial condition for a
scroll wave. The $u$-field is plotted.  Hitting the space bar will
start the simulation. This is a coarse resolution run showing the
speed possible with EZ-Scroll simulations.  With the pointer in the
EZ-Scroll window, you can:

\begin{list}{dummy}
{\labelwidth=0.2in\labelsep=4pt\partopsep=-10pt\parsep=-10pt\itemsep=12pt}
\item[(1)] Switch between $u$-field, $v$-field, and no field by typing
{\em u}, {\em v}, or {\em n} respectively.

\item[(2)] toggle filament plotting by typing {\em f}.  (The filament
is just the intersection of 2 contours.)

\item[(3)] Toggle the clipping plane by typing {\em c}.

\item[(4)] Pause the simulation by typing {\em p}, and resume by
typing a space.

\item[(5)] Rotate the image by first pausing the simulation, then by
holding down the left mouse button and moving the cursor.

\item[(6)] The key {\em r} resets the view to the initial (start up)
view, and {\em z} sets the view to looking down the $z$-axis with the
$x$- and $y$-axes in the usual orientation.  This view is useful for
moving the scroll.

\item[(7)] The arrow keys move the scroll in the x-y directions. The
$+$ key moves the scroll up the z-axis and the $-$ moves it
down the z-axis.  Again, for moving the scroll it is best first to
have set the view by typing {\em z}.

\item[(8)] Stop the simulation by typing: \\
{\em q} for soft termination with all files closed or  \\
{\em ESC} for immediate termination without writing final
conditions (equivalent to typing control-C from the shell). 

\end{list}

After a successful run, you will have a file {\em fc.dat} in your
directory which contains the final conditions of the run. If you copy
this file to {\em ic.dat}, then the next time you run {\em ezscroll},
this file will be read and used as an initial condition.

{\bf IV. Equations and Parameters} 

The model reaction-diffusion equations are [1,2]:
$$
\frac{\partial u }{ \partial t} = \nabla^2 u + 
	\epsilon^{-1} u(1-u)(u-u_{th}(v)), ~~~
\frac{\partial v }{ \partial t} = D_v \nabla^2 v + g(u,v)
$$
The method employed in EZ-Scroll is (essentially) independent of the
choice of the functions $u_{th}(v)$ and $g(u,v)$ and the user is free
to set these to whatever is desired.  See {\em ezstep3d.h}.

In the simplest case
$$
u_{th}(v) = {{v+b}\over{a}}, ~~~
g(u,v) = u-v, 
$$ 
so that $a,b,$ and $\epsilon$ are parameters of the reaction kinetics.
$D_v$ is the ratio of diffusion coefficients ($D_u \equiv 1$ by choice
of length scales). In addition, there are lengths specifying the
simulation volume: $L_x, L_y$, and $L_z$.  Of these I choose to
specify only $L_x$ and let $L_y$ and $L_z$ be determined from the
number of grid points (see below).  Thus the ``physical'' parameters
in the simulation are: {\sf a, b, 1/epsilon, Lx} and {\sf Dv}.

The ``numerical'' parameters for the simulation are: {\sf nx, ny, nz}
= number of spatial grid points in each direction, {\sf ts} = time
step as fraction of the diffusion stability limit, and {\sf delta} =
small numerical parameter [1,2].  From {\sf nx} and {\sf Lx} the grid
spacing is determined and from this and {\sf ny, nz} the lengths {\sf
Ly, Lz} are determined.

The other parameters set in
{\em task.dat} are:

\begin{list}{dummy}
{\labelwidth=0.1in\labelsep=0pt\partopsep=-10pt\parsep=-10pt\itemsep=0pt}
\item[~]{\sf Number of time steps to take} \\
        {\sf Time steps per plot.} Also set the number of time
        steps per filament computation. \\
        {\sf write\_filament.} Flag for writing filament data \\
        {\sf Time steps per write to history file} \\
        $(i,j, k)$ {\sf history point} \\ 
	\\
	{\sf initial field} \\
	{\sf initial condition type.}  \\
	{\sf simulation and rotation resolutions} \\
	{\sf output type} \\
	\\
	{\sf verbose} 
\end{list}

These are more or less self-explanatory.  
If {\sf write\_filament} is non-zero then the filament will be
computed every {\sf Time steps per plot} (whether or not there is any
graphics) and the filament data will be written to a file ({\em
filament.dat}).  Note: each line of filament data consists of the
time and a pair a points defining a line segment on the filament. 

If the $(i,j,k)$ point is in the domain, then a time series at the
$(i,j,k)$ grid point will be saved every {\sf Time steps per write}
assuming it is not zero.  

I leave it to you to look at different initial condition types at 
the end of {\em ezscroll.c}.

If {\sf simulation resolution}=1, then all the points in the
simulation grid are used for the iso-surface and filament computation.
If {\sf simulation resolution}=2, then every other point in each
direction is used, i.e.~1/8 as many points.  This results in faster
graphics.  The same applies to the {\sf rotation resolution} for
setting the resolution during rotations only.  Note: the {\sf
simulation resolution} sets the resolution of filament computation
even in the absence of any graphics, so for accurate filament data set
{\sf simulation resolution} to~1.

The other place to look for ``parameters'' is in the header files.
The main compilation parameters are in {\em ezscroll.h}.  Note that
{\sf SPLIT} is 1 in the distribution, but I think 0 is a better choice
except at large time steps (See {\em ezstep3d.c}).  Many of the macro
definitions in the other header files can be replaced with variables.

\bigskip

{\bf References} 

[1] D. Barkley, M. Kness, and L.S. Tuckerman, Phys. Rev. A {\bf 42}, 2489 (1990).

[2] D. Barkley, Physica {\bf 49D}, 61 (1991).

[3] D. Barkley, Phys. Rev. Lett {\bf 68}, 2090 (1992).

[4] D. Barkley, Phys. Rev. Lett. {\bf 72}, 164 (1994).

[5] M. Dowle, R.M. Mantel, and D. Barkley, 
Int.~J.~Bif.~Chaos {\bf 7}, 2529 (1997).

Please send comments to barkley@maths.warwick.ac.uk

\end{document} 
