%-----------------------------------------------------------------------------%
%-----------------------------------------------------------------------------%
%                                                                             %
%                                 EZ-SPIRAL                                   %
%                     A Code for Simulating Spiral Waves                      %
%                                                                             %
%                                 Version 3                                   %
%                                                                             %
%                                                                             %
% $Revision: 3.1.1.1 $
% $Date: 2002/08/01 08:21:56 $
% 
%-----------------------------------------------------------------------------%
%-----------------------------------------------------------------------------%

% This is a LaTex source file.

\documentclass[12pt]{article}

\textheight 9in  \textwidth 6.5in
\topmargin -.7in \oddsidemargin .0in

\begin{document}
\baselineskip = 20pt plus 1pt minus 1pt
\parskip = 8pt
\parindent=0pt

\begin{center} 
{\bf EZ-SPIRAL DOCUMENTATION} 
\end{center} 

\bigskip
{\bf I. Philosophy of EZ-Spiral} 
\smallskip

EZ-Spiral is a simplified version of the C program that I use to study the
dynamics of spiral waves in excitable media [1-4,6]. EZ-Spiral is designed to
be as simple as possible while offering what I believe to be the most
important features for spiral simulations. These include:
\begin{list}{dummy}
{\labelwidth=0.2in\labelsep=4pt\partopsep=-10pt\parsep=-10pt\itemsep=12pt}
\item[(1)] Continuous range of spatio-temporal resolutions.
\item[(2)] Choice of implicit or explicit time stepping of the reaction 
           kinetics. 
\item[(3)] Choice of ratio of diffusion coefficients.
\item[(4)] Computation of spiral-tip location.
\item[(5)] Automatic generation of initial conditions.
\item[(6)] Interactive graphics with control of spiral location.
\end{list}

The code should be of use to those who are interested in the serious
study of excitable media as well as those who simply want to generate
pretty pictures. Most of the documentation is contained in the code
itself. The notation for the model reaction-diffusion equations and a
description of parameters are given in Sec. IV.

{\bf Changes in Version 2.}  Version 2 of EZ-Spiral fixes many of the
awkward properties of Version 1.  In particular, almost all parameters
one routinely wants to change are now contained in {\em task.dat} and
are read in at run time.  This eliminates most of the recompiling that
was necessary with Version 1.  The other major change is that the
graphics subroutines have been rewritten using X11 calls and hence the
code is portable to a large number of machines.  Elementary graphics
are even available for black and white X-terminals.  Finally, parts of
the code have been cleaned up and simplified.  By editing just two
lines it is now possible to simulate a large variety of models with
EZ-Spiral (see Sec. IV).  I include at the end (Sec. V) a simple
recipe for obtaining turbulent spirals.  Note that the format of {\em
task.dat} has changed slightly from Version 1.  The format of the
field files has not changed.

{\bf Changes in Version 2.3} The primary change in Version 2.3 is that
the the higher-order implicit $u$ time stepping is implemented.  See [6].

{\bf Changes in Version 3} Version 3 comes from a large re-write of the
code. It in fact was obtained from reducing EZ-Scroll to the 2D case. I
decided to keep separate 2D and 3D version of the code. The main new features
are that the 2D domain can now be a rectangle and the graphics is heavily
OpenGL based, with primitive X11 calls only used for opening the window and
for event handling. The format of {\em task.dat} has changed from Version 2.

% \pagebreak

\medskip
{\bf II. Running EZ-Spiral} 
\smallskip

{\bf Files:} You should have the following files: {\em ezspiral.c, ezstep.c,
eztip.c, ezgraphGL.c, ezspiral.h, ezstep.h ezgraphGL.h, task.dat}, and {\em
Makefile}. You will probably want to save copies of these files (in compressed
tar format).

{\bf make}: It is up to you to edit {\em Makefile} as necessary for your
system. You need an ANSI C compiler and for graphics you need X11 and
OpenGL. If you run on an up-to-date Linux system you should have everything
you need. Typing {\em make} will make {\em ezspiral}.  If you edit one of the
source or header files you must remake EZ-Spiral.

{\bf Running:} Make {\em ezspiral} by typing {\em make}. Then run by typing
{\em ezspiral}. A window should open. With the pointer in the window type a
{\em space} on the keyboard. A spiral should be generated.  The $v$-field is
plotted. This is a coarse resolution run showing the speed possible with
EZ-Spiral simulations.  With the pointer in the EZ-Spiral window, you can move
the spiral around using the arrow keys on the keyboard.  Center the spiral in
the box.

A simulation will finish when either:

\begin{list}{dummy}
{\labelwidth=0.1in\labelsep=0pt\partopsep=-10pt\parsep=-10pt\itemsep=0pt}
\item[~]the number of time steps set in {\em task.dat} is taken, \\
	or you type {\em q} or {\em Q} with the pointer in the spiral 
        window. \\
	or you type {\em Esc} with the pointer in the spiral window \\
	or you use the menu produced by the window manager (assuming there
        is one) to kill the graphics window.
\end{list}
The last two possibilities will result in a hard exit with nothing saved.

After a successful run with a soft exit (the first two cases above), you will
have a file {\em fc.dat} in your directory which contains the final conditions
of the run (either at the final time or at the type you stopped the simulation
with {\em q} or {\em Q}). If you copy this file to {\em ic.dat}, then the next
time you run {\em ezspiral}, this file will be read and used as an initial
condition.

To get some insight into what the code can do, try the following runs
in order:
\begin{enumerate}

\item
In {\em task.dat}, change {\sf nx} and {\sf ny} from 61 to 121, change {\sf
Time steps per plot} to 16, and change {\sf verbose} from 2 to 1 (the normal
value). Copy {\em fc.dat} from the run with {\sf nx}={\sf nx}=61 to {\em
ic.dat}. Run {\em ezspiral}.  This is a more accurate run and is getting close
to a fully resolved simulation of the underlying PDEs.

\item
While the code is running put the pointer in the spiral window.  Typing either
{\em p} or {\em P} within the spiral window will pause the simulation until
you type a {\em space} on the keyboard.  Typing {\em t} will toggle tip
plotting.  Typing {\em u} will show the $u$-field rather than the
$v$-field. You can go back to the $v$-field by typing {\em v}.  Typing {\em n}
will show no field. You can, however, still plot the tip path if you desire.
{\em Note that for coarse simulations, tip finding can produce uncertain
results and will probably fail.}

\item
In {\em task.dat} change {\sf write\_tip} from 0 to 1 and {\sf time steps per
write to history file} to 10.  Now run run {\em ezspiral} again.  After this
run you will have a file {\em tip.dat} with the path of the spiral tip $(t,
x_{tip}, y_{tip})$, and a file {\em history.dat} with the time series of
values $(t,u,v)$ at the grid point $(10,20)$.  You can set the history point
to some other value in {\em task.dat}.

\item
Change the kinetics parameter {\sf a} in {\em task.dat} from 0.75 to 0.62.
Run again. With these parameters the spiral will ``meander''. You can see this
best by plotting the tip.  


\end{enumerate}
You should have the hang of it by now.

{\bf III. Further Comments on EZ-Spiral} 

It is expected that you will modify the code as needed for your
purposes.  The code should be more or less self-explanatory. The bulk
of the code is devoted to I/O, graphics, and tip finding. Almost all
of the execution time is spent on just a few lines of code in the
routine {\sf step()} in {\em ezstep.c}.

Note that I have used many external (i.e.~global) variables and have
not used any structures.  I thought that this would make the code more
readable, and therefore usable, to those not entirely familiar with C.
I believe that the crucial {\sf for} loops in routine {\sf step()}
compile to very efficient code. See comments in {\em ezstep.h}.

I make no claims as to what constitutes a fully resolved simulation of
the underlying PDEs.  If you are concerned about this I would
recommend having at least 6 points across the interface (black in the
$u$-field plot) and a ratio of {\sl dt}/$\epsilon<0.5$.  You might
also decrease {\sf delta} by an order of magnitude or more.  You
should experiment for yourself.  Note that for {\sl dt}/$\epsilon<1$
you can use either the explicit or the implicit form of the reaction
kinetics.  The explicit form is slightly faster, but the new implicit
form [6] is definitely more accurate and I highly recommend it.  For
{\sf delta} very small, you should remove the {\sf if} from the {\sf
for} loops in {\sf step()}.

In general I believe in the 9-point formula for the Laplacians [1,6],
rather than the 5-point formula.  The 9-point formula maintains
rotational symmetry to leading order.  For coarse-grained simulations
with $D_v=0$ the difference in execution time is small compared with
what is gained.  If you are unconvinced, run the simulation described
in (1) with both 5-point and 9-point Laplacians.

I believe tip finding is quite reliable for well resolved simulations.
For coarse-grained simulations it may fail miserably.

With {\sf GRAPHICS} in {\em ezspiral.h} set to 0, none of the X11
graphics is compiled and this should allow the code to run on {\em
any} machine.  However, do not run EZ-Spiral on a vector machine
without appropriate modification.

{\bf IV. Equations and Parameters} 

The model reaction-diffusion equations are [1,2]:
$$
\frac{\partial u }{ \partial t} = \nabla^2 u + 
	\epsilon^{-1} u(1-u)(u-u_{th}(v)), ~~~
\frac{\partial v }{ \partial t} = D_v \nabla^2 v + g(u,v)
$$
where in the simplest case
$$
u_{th}(v) = {{v+b}\over{a}}, ~~~
g(u,v) = u-v, 
$$ 
so that $a,b,$ and $\epsilon$ are parameters of the reaction
kinetics.  $D_v$ is the ratio of diffusion coefficients ($D_u \equiv
1$ by choice of length scales). In addition, $L_x$ and $L_y$ are the lengths
of the sides. Thus the ``physical'' parameters in the
simulation are: $a, b, \epsilon, L_x, L_y$, and $D_v$.

The method employed in EZ-Spiral is (essentially) independent of the
choice of the functions $u_{th}(v)$ and $g(u,v)$ and the user is free
to set these to whatever is desired.  See {\em ezstep.h}.

The ``numerical'' parameters for the simulation are: {\sf nx}, {\sf ny}, {\sf
ts} (= time step as fraction of the diffusion stability limit), and {\sf
delta} (= small numerical parameter [1,2]).  The other parameters set in {\em
task.dat} are more or less self-explanatory.

Depending on the hardware you have, plotting can take a significant amount of
time, so choose {\sf Time steps per plot} judiciously.  Also note that
plotting the $u$-field is faster than plotting the $v$-field. See the code.

{\bf V. Spiral Turbulence with EZ-Spiral} 

I include in {\em ezstep.h} two choices for the slow kinetics $g(u,v)$
other than the choice $g(u,v) = u-v$.  These are $g(u,v) = u^3-v$ and
the function given by B\"ar and Eiswirth[5].  Both choices produce
spiral breakup followed by spiral ``turbulence''.  Here I give a
simple recipe for spiral turbulence.

In {\em ezstep.h} turn off the ``standard model'' by putting {\sf \#if
0} and turn on the ``simple model for turbulent spirals'' by putting
{\sf \#if 1}.  Remake ezspiral.  Then in {\em task.dat} set: {\sf
a=0.75, b=0.06, 1/eps=12.0, L=80, Dv=0, nx=ny=121, ts = 0.8}, and {\sf
delta=1.e-4}.  Set the graphics parameters to your choice.  Now run
ezspiral.  You should see spiral breakup followed by spiral
turbulence. This simulation is only qualitatively correct.  For a more
quantitative and convincing example, set {\sf 1/eps=14} and {\sf
ndim=241} (and all other parameters as before).  This time you will
see the formation of a single spiral.  After running sufficiently long
that you are convinced that the spiral is stable, stop the simulation
and move {\em fc.dat} to {\em ic.dat}.  Set {\sf 1/eps=12} and
rerun.  You will now see the spiral breakup.  You can increase the
spatial resolution still further if you desire and you can take
several small steps in {\sf 1/eps} if you wish, but I believe you will
get essentially the same result.


{\bf References} 

[1] D. Barkley, M. Kness, and L.S. Tuckerman, Phys. Rev. A {\bf 42}, 
`2489 (1990).

[2] D. Barkley, Physica {\bf 49D}, 61 (1991).

[3] D. Barkley, Phys. Rev. Lett {\bf 68}, 2090 (1992).

[4] D. Barkley, Phys. Rev. Lett. {\bf 72}, 164 (1994).

[5] M. B\"ar and  M. Eiswirth, Phys. Rev. E {\bf 48}, 1635 (1993).

[6] M. Dowel, R.M. Mantel and D. Barkley, Int. J. Bif. Chaos, 7(11) 2529--2546
(1997).

Please send comments to barkley@maths.warwick.ac.uk

\end{document} 
